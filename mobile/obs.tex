- criando o projeto --> npx create-expo-app --template

- iniciando o projeto --> npx expo start

- mapeamento de importações, --> npm install --save-dev babel-plugin-module-resolver
- configuração "babel.config.js e tsconfig.json"

- instalação das fontes"
- npx expo install expo-font @expo-google-fonts/roboto
- importa no app -> import {useFonts, Roboto_400Regular, Roboto_700Bold } from '@expo-google-fonts/roboto'
- configuraçao do estado que carrega as fontes configuradas const [fontsLoaded] = useFonts([ Roboto_400Regular, Roboto_700Bold]);


- instalando a Library "Native Base" => ganha muito na produtividade, componentes prontos
- npm install native-base
- instalando react-native-svg => trabalhar com svg
- npx expo install react-native-svg@12.1.1
- instalando react-native-safe-area-context
- npx expo install react-native-safe-area-context@3.3.2

- importa no app -> import { NativeBaseProvider } from 'native-base'
- definição do thema da aplicação "theme/index.ts "modifica o thema padrão do Native Base"
- Configuração do thema no app da aplicação usando o 'NativeBaseProvider do NativeBase'
- Coloca o NativeBaseProvider em volta da aplicação com a definição do theme do arquivo theme/index.ts
<NativeBaseProvider theme={THEME}>
...
</NativeBaseProvider>

- Trabalhando com imagens
- definição das typagems para usar imagens na aplicação '@types/png.d.ts'.

- Trabalhando com svg
- instale outra biblioteca para trabalhar o svg como componentes
- npm i react-native-svg-transformer --save-dev
- configuração da biblioteca no arquivo metro.config.js => responsavel para podemos utilizar o svg na mesma sintaxe de um componente
- definição das typagems para usar imagens na aplicação '@types/svg.d.ts'.

- Navegação
- Stack Navigator  "Cora da aplicação" -> npm install @react-navigation/native
- dependencias que precisa lidar com a parte da navegação e suguras da aplicação -> npx expo install react-native-screens react-native-safe-area-context
- extrategia de naveção que será utilizada de forma publica-> npm install @react-navigation/native-stack
- extrategia de navegação que será utilizada de forma privada => npm install @react-navigation/bottom-tabs

routes/auth.routes
routes/app.routes
index.tsx

- Routes
- 2 rotas Publica e Privada
- 2 typagem para rota de aplicação, não faça a typagem generica "navegation.d.ts"
- type AuthRoutes = {
    signIn: undefined;
    signUp: undefined;
}
export type AuthNavigatorRoutesProps = NativeStackNavigationProp<AuthRoutes>;

const { Navigator, Screen } = createNativeStackNavigator<AuthRoutes>();

- type AppRoutes = {
    home: undefined;
    history: undefined;
    profile: undefined;
    exercice: undefined;
};

export type AppNavigatorRoutesProps = BottomTabNavigationProp<AppRoutes>;

const { Navigator, Screen } = createBottomTabNavigator<AppRoutes>();


-selecionando a foto do usuario e editando
- npx expo install expo-image-picker
-adicionar o plugin de configuração no app.json

"plugins": [
      [
        "expo-image-picker",
        {
          "photosPermission": "The app accesses your photos to let you share them with your friends."
        }
      ]
    ],

buscando informações do arquivo selecionando 'foto'
npx expo install expo-file-system


formularios
React-hook-form
- npm install react-hook-form

(control, handleSubmit, formState {erros })

-Usa o useForm() para acessar o "control" passado para todos os inputs sendo controlados no mesmo formulario,
-Controller do react-hook-form fica responsavel para amarzenar os valores do estado atravez da proprieda 'field'
-handleSubmit ela passa todo o conteudo do formulario para outra funcão de amarzenamento
-defaultValues -> define valores iniciais


regras nos formularios
-rules:
<Controller
                        control={control}
                        name='email'
                        rules={{
                            required: 'Informe o e-mail',
                            pattern: {
                                value:/^[A-Z0-9._%+-]+@[A-Z0-9.-]+\.[A-Z]{2,}$/i,
                                message: 'E-mail inválido'
                              }
                        }}
...
</Controller>                        

-formState: { errors } expoe os erros para o usuario na aplicação

ou podemo usar 2 bibliotecas 
resolvers -> integração do react com o hookform
yup -> bibioeta de validação
npm install @hookform/resolvers yup


=============================back-end=================================
=> pasta api
=> npm run dev, roda o servidor na porta 3333

instalação do aplicativo "beekpeer studio" visualização das tabelas

=> criando um novo usuario
- usando o fetch, function Asycn Await do Java Script 
// function handleSignUp({ name, email, password }: FormDataProps) {
    async function handleSignUp({ name, email, password }: FormDataProps) {
    // fetch('http://192.168.0.16:3333/users', {
       const response = await fetch('http://192.168.0.16:3333/users', {
            method: 'POST',
            headers: {
                'Accept': 'application/json',
                'Content-Type': 'application/json',
            },
            body: JSON.stringify({ name, email, password })
        })

        const data = await response.json();
        console.log(data)
            
        // .then(response => response.json())
        // .then(data => console.log(data))
    }

=> usando a bibioeta Axios 'cliente http' fazer todas as requisições
=> npm i axios

-> cria a pasta services/api.ts
-> define as configurações do axios

const api = axios.create({
    baseURL: 'http://192.168.0.16:3333'
});

=> criando um novo usuario
- usando o fetch, function Asycn Await do Java Script 

async function handleSignUp({ name, email, password }: FormDataProps) {
    const response = await api.post('/users', { name, email, password });
    console.log(response.data)
}

- tratando erros
async function handleSignUp({ name, email, password }: FormDataProps) {
        try{
            const response = await api.post('/users', { name, email, password });
            console.log(response.data)
        } catch (error) {
            if(axios.isAxiosError(error)) {
                Alert.alert(error.response?.data.message);
            }
        }
}

- tratando erros usando o interceptors
- criar a pasta utils/AppError.ts define as configurações
- na pasta services/Api.ts define as configuraçoes do interceptors

api.interceptors.response.use((response) => response, (error) => {
    if(error.response && error.response.data) {
      return Promise.reject(new AppError(error.response.data.message))
    } else {
      return Promise.reject(error)
    }
});

Contexto compatilhar dados entre os componentes
-cria pasta contexts/AuthContext.tsx

import { Children, ReactNode, createContext } from 'react'

export const AuthContext = createContext({})

-configura o arquivo no app.tsx
import { AuthContext } from '@contexts/AuthContext';

  return (
    <NativeBaseProvider theme={THEME}>
        ...
        <AuthContext.Provider value={{
          id: '1',
          name: 'lauande',
          email: 'lauande@lauande.com.br',
          avatar: 'rodrigolauande.png'
        }}>
          {fontsLoaded ? <Routes/> : <Loading /> }
        </AuthContext.Provider>
    </NativeBaseProvider>
  );
}

-configura no arquivo routes/index.tsx

const contextData = useContext(AuthContext)
console.log('usuario logado', contextData)

pronto consigo ter acesso aos dados

=> tipagem DTO
-cria a pasta dtos/UserDTO.ts, define e exporte as tipagems

=> tipagem DTO no arquivo AuthContext.tsx
import { UserDTO } from '@dtos/UserDTO'

export type AuthContextDataProps = {
    user: UserDTO
}

export const AuthContext = createContext<AuthContextDataProps>({} as AuthContextDataProps)

=> define a tipagem no app.tsx
        <AuthContext.Provider value={{
          user: {
            id: '1',
            name: 'lauande',
            email: 'lauande@lauande.com.br',
            avatar: 'rodrigolauande.png'
          }
        }}>
          {fontsLoaded ? <Routes/> : <Loading /> }
        </AuthContext.Provider>

        agora você tem as tipagens dos arquivos 'user'


=> refatoração do Context()
-> cria uma nova tipagem e uma nova função para ser exportada no arquivo AuthContext.tsx

import { UserDTO } from '@dtos/UserDTO'
import { Children, ReactNode, createContext } from 'react'

export type AuthContextDataProps = {
    user: UserDTO
}

type AuthContextProviderProps = {
    children: ReactNode;
}

export const AuthContext = createContext<AuthContextDataProps>({} as AuthContextDataProps)

export function AuthContextProvider ({children}: AuthContextProviderProps) {
    return (
        <AuthContext.Provider value={{
            user:  {
              id: '1',
              name: 'lauande',
              email: 'lauande@lauande.com.br',
              avatar: 'rodrigolauande.png'
            }
          }}>
            {children}
          </AuthContext.Provider>
    )
}

no arquivo app.tsx refatore o AuthContext.Provider usando a nova função criada no arquivo Auth.Context.tsx

"AuthContextProvider"

=> criando o proprio "hook"
cria uma nova pasta 'hooks/useAuth.tsx

import {useContext} from 'react
import {AuthContext} from '@context/AuthContext'

export function useAuth() {
  const context = useContext(AuthContext)

  return context
}

a função vai pegar toda a tipagem definida no AuthContext => User = UserDTO

dentro do arquivo routes.index.tsx declara a nova função criada

import {useAuth} from '@hooks/useAuth'

const { user } = useAuth()

=> Compartilhando estado e atualizando o estado dentro do Context()
-no arquivo AuthContext.tsx redefina a função AuthContextProvider()
-colocando as informações do userDTO em um estado 'user'
export function AuthContextProvider ({children}: AuthContextProviderProps) {
    const [ user, setUser ] = useState({
        id: '1',
        name: 'Rodrigohenriquelauande',
        email: 'lauande@lauande.com.br',
        avatar: 'rodrigolauande.png'
    })

    return (
        <AuthContext.Provider value={{ user, setUser }}>
            {children}
          </AuthContext.Provider>
    )
}

-no arquivo screens/signIn atualize a função handleLogin()
function handleLogin({email, password}: FormDataProps) {
        setUser({
            id: '',
            name: '',
            email,
            avatar: '',
        })
    }


=>pode ser melhorado centralizando dentro do contexto criando uma nova função signIn()
AuthContext.tsx
AuthContextProvider()
function singIn(email: string, password: string) {
        setUser({
            id: '',
            name: '',
            email,
            avatar: '',
        })
}

screens/signIn
function handleLogin({email, password}: FormDataProps) {
        singIn(email, password)
    }

=> buscando dados do usuario
- AuthContext.tsx
- AuthContextProvider

export type AuthContextDataProps = {
    user: UserDTO
    singIn: (email: string, passoword: string) => Promise<void>
}

async function singIn(email: string, password: string) {
        try {
            const { data } = await api.post('/sessions', { email, password });
            
            if (data.user) {
                setUser(data.user)
            }
        } catch (error) {
            throw error
        }
}

=> redirecionamento do usuario logado
routes/index.tsx

const { user } = useAuth()
console.log('USUÁRIO LOGADO', user)

<Box flex={1} bg='gray.700'>
<NavigationContainer theme={theme}>
    {user.id ? <AppRoutes/> : <AuthRoutes />}
</NavigationContainer>
</Box>

=> AsyncStorage
- persistir as informações do usuario logado na aplicação
npx expo install @react-native-async-storage/async-storage

=> criar o arquivo storage/storageConfig.ts
const USER_STORAGE = '@gymignite:user';

export { USER_STORAGE };

=> criar o arquivo storage/storageUser.ts
import AsyncStorage from "@react-native-async-storage/async-storage";

import { UserDTO } from '@dtos/UserDTO';
import { USER_STORAGE } from "./storageConfig";

export async function storageUserSave(user: UserDTO) {
    await AsyncStorage.setItem(USER_STORAGE, JSON.stringify(user));
}

export async function storageUserGet() {
    const storage = await AsyncStorage.getItem(USER_STORAGE);

    const user: UserDTO = storage ? JSON.parse(storage) : {};

    return user;
}


=> usa a função storageUserSave() e storageUserGet() no arquivo AuthContext.tsx
async function singIn(email: string, password: string) {
        try {
            const { data } = await api.post('/sessions', { email, password });
            
            if (data.user) {
                setUser(data.user)
                storageUserSave(data.user)
            }
        } catch (error) {
            throw error
        }
    }

    async function loadUserData() {
        const userLogged = await storageUserGet();

        if (userLogged) {
            setUser(userLogged)
        }
    }

    useEffect(() => {
        loadUserData()
    },[]) 

=> Refresh token - atualizar o token da aplicação